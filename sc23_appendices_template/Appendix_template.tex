% 
\documentclass[sigconf,nonacm=true]{acmart}
%\documentclass[sigconf]{acmart}
\usepackage{fancyhdr}
\usepackage{fvextra}
\usepackage[utf8]{inputenc}

% tools for moving from verbatim to tttext
\usepackage{hanging}

% make a command to allow line breaks in urls
% from https://tex.stackexchange.com/questions/315369/how-to-deal-with-bad-line-wrapping-of-texttt
\newcommand*\ttvar[1]{\texttt{\expandafter\dottvar\detokenize{#1}\relax}}
\newcommand*\dottvar[1]{\ifx\relax#1\else
  \expandafter\ifx\string/#1\string/\allowbreak\else#1\fi
  \expandafter\dottvar\fi}


% The "title" command has an optional parameter, allowing the author to define a "short title" to be used in page headers.
\title[FTIO: Detecting I/O Periodicity Using Frequency Techniques]{Appendix: Artifact Description/Artifact Evaluation}
\thispagestyle{empty}

\begin{document}
\setcopyright{acmcopyright}
\settopmatter{printfolios=false, printccs=false, printacmref=false}
\copyrightyear{2023}
\acmYear{2023}
\acmConference[ProTools '23]{Workshop on Programming and Performance Visualization Tools}{November 12--17, 2023}{Denver, CO, USA}
\acmBooktitle{Workshop on Programming and Performance Visualization Tools(ProTools '23), November 12--17, 2022, Denver, CO, USA}
% \acmPrice{15.00}
% \acmDOI{10.1145/3295500.3356138}
% \acmISBN{978-1-4503-6229-0/19/11}
\acmPrice{}
\acmDOI{}
\acmISBN{}

\sloppy
\maketitle


% By default, the full list of authors will be used in the page headers. Often, this list is too long, and will overlap
% other information printed in the page headers. This command allows the author to define a more concise list
% of authors' names for this purpose.
\renewcommand{\shortauthors}{ Tarraf, et al. }

\section*{Artifact Identification}
%START_LATEX
In the submitted paper, an approach was demonstrated 
that allows for predicting (online) ...

To demonstrate and evaluate our approach, we included results from two experimental campaigns:
(1) ...
(2) ...

\paragraph{Experimental runs at ...:} 
For the paper, we performed the experimental runs (Sections 2, 3, 4, and 7) using several benchmarks ....

\paragraph{Semi-synthetic traces:} 
...

\section*{Reproducibility of Experiments}
In the next two sections, we explain in detail the steps involved in creating all the results shown in the paper. 
Both types of experiments...

\section{Experimental runs on ...}
\label{sec:exp}
Through Sections 2, 3, 4, and 7 of the paper, 
....

\subsection{Workflow description}
....

\subsection{Time needed for the workflow}
....

\subsection{The experiments and their results}
.....


%%%%%%%%%%%%%%%%%%%%%%%%%%%%%%%%%%%%%%%%%%%%%%%%%%%%%%%%%%%%%%%%%%%%%%%%
\section{Evaluation with semi-synthetic traces}
\label{sec:tra}
...
\end{document}
