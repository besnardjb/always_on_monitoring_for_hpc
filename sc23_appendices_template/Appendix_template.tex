% 
\documentclass[sigconf,nonacm=true]{acmart}
%\documentclass[sigconf]{acmart}
\usepackage{fancyhdr}
\usepackage{fvextra}
\usepackage[utf8]{inputenc}
\usepackage{xcolor}
\usepackage{listings}
\lstset{basicstyle=\ttfamily,
  showstringspaces=false,
  commentstyle=\color{red},
  keywordstyle=\color{blue},
  morekeywords={tau_profile_inspect},
  backgroundcolor=\color{lightgray},
  frame=single,
}

% tools for moving from verbatim to tttext
\usepackage{hanging}

% make a command to allow line breaks in urls
% from https://tex.stackexchange.com/questions/315369/how-to-deal-with-bad-line-wrapping-of-texttt
\newcommand*\ttvar[1]{\texttt{\expandafter\dottvar\detokenize{#1}\relax}}
\newcommand*\dottvar[1]{\ifx\relax#1\else
  \expandafter\ifx\string/#1\string/\allowbreak\else#1\fi
  \expandafter\dottvar\fi}


% The "title" command has an optional parameter, allowing the author to define a "short title" to be used in page headers.
\title[On Enabling Continuous Optimization for HPC Workloads]{Appendix: Artifact Description/Artifact Evaluation}
\thispagestyle{empty}

\begin{document}
\setcopyright{acmcopyright}
\settopmatter{printfolios=false, printccs=false, printacmref=false}
\copyrightyear{2023}
\acmYear{2023}
\acmConference[ProTools '23]{Workshop on Programming and Performance Visualization Tools}{November 12--17, 2023}{Denver, CO, USA}
\acmBooktitle{Workshop on Programming and Performance Visualization Tools(ProTools '23), November 12--17, 2022, Denver, CO, USA}
% \acmPrice{15.00}
% \acmDOI{10.1145/3295500.3356138}
% \acmISBN{978-1-4503-6229-0/19/11}
\acmPrice{}
\acmDOI{}
\acmISBN{}

\sloppy
\maketitle


% By default, the full list of authors will be used in the page headers. Often, this list is too long, and will overlap
% other information printed in the page headers. This command allows the author to define a more concise list
% of authors' names for this purpose.
\renewcommand{\shortauthors}{ JB. Besnard and A. Tarraf }

\section*{Artifact Identification}
%START_LATEX
In the submitted paper, an approach was demonstrated that performs always-on monitoring. To demonstrate and evaluate our approach, 
different experiments were executed with Nek5000 and IMB-IO. In this appendix,we will 
handle: 
(1) how to measure library wrapper overhead,
(2) how to start the metric proxy and its overhead, 
% (2) how to visualize the traces with Prometheus and Grafana, and finally
(3) how to generate performance models with Extra-P. 
Moreover, we will also have a quick look at the file format and parsing done. 
Additionally, for each step, we will examine the workflow, time needed and the relevance 
of the experiments and results.

\section*{Reproducibility of Experiments}
In the next two sections, we explain in detail the steps involved in creating all the results shown in the paper while 
at the same time focusing on the aspects previously listed. In general, all 
the results presented in the paper are reproducible. Moreover, 
the approach is straightforward and can be easily executed 
on different systems. All the relative data are providedin \url{https://github.com/besnardjb/always\_on\_monitoring\_for\_hpc/tree/main}.

%%%%%%%%% wrappers

\section{Measuring Library Wrapper Cost}
\label{sec:wrapper}

This section details how we made the PMPI function call overhead measurement. It refers to files located in \texttt{wrapper\_overhead}.
\subsection{Workflow description}

The main steps to reproduce our results are:

\begin{itemize}
\item Build the benchmarks (make)
\item Run the measurement script.
\item Plot the results
\end{itemize}

\subsection{Time needed for the workflow}

As we are dealing with very small durations, we had to set a large number of averages(10 billions) to stabilize results. Running these measurements then requires several hours. If you change the number of iterrations in \texttt{iterate.c}, you may get less stable results. To run the measurements you may run the \texttt{measure\_interpos.sh} script.

\subsection{The experiments and their results}

\paragraph*{\textbf{Figure 5:}} This figure presents the wrapping overhead in function of the scope of the wrapper function, respectively static, another transtation unit in the same binary and in a library. It shows that a small overhead is added when the function is in a library, mainly due to the PLT jump. Artifacts associated with these results are provided in \texttt{wrapper\_overhead/PAPER\_DATA}, you may enter this directory and regenerate the plot with \texttt{python3 ../plot\_interpos.py}.

\paragraph*{\textbf{Figure 6:}} This figure presents results similar to Figure 5 with the exception that it is in time (instead of relative overhead). The goal is to allow a quantitative measurement of this difference of a few nanoseconds. Artifacts associated with these results are provided in \texttt{wrapper\_overhead/PAPER\_DATA}, you may enter this directory and regenerate the plot with \texttt{python3 ../plot\_interpos\_time\_avg.py}. Note that the average instrumented cost for library (first collumn in \texttt{lib.dat}) is used as baseline for the proxy overhead in section \ref{sec:monitor}.

%%%%%%%%%%%%%%%%%%%%%%%%%%%%%% Metric proxy
\section{Always-on monitoring}
\label{sec:monitor}

This section describes how to perform always on monitoring using the metric proxy. We first provide means of generating the data for the paper.
In addition, we provide the raw data used for the paper figures for reference.

\subsection{Workflow description}

The main steps to reproduce our results are:

\begin{itemize}
\item Install the metric proxy
\item Run the measurement script
\end{itemize}

\subsubsection{Install the metric proxy}

The metric proxy is provided in the \texttt{metric\_proxy} subdirectory. Installing it in a given prefix is a matter of doing:

\begin{lstlisting}[language=bash]
cd ./metric_proxy/
mkdir BUILD && cd BUILD
../configure --prefix=$HOME/metric_proxy
make -j8 install
\end{lstlisting}

Note the following requirements:

\begin{itemize}
\item MPI C compiler in the path
\item Python in the path
\end{itemize}

\subsubsection{Run measurements}

The script (located in \texttt{./proxy\_overhead}) will run a simple code calling \texttt{MPI\_Comm\_rank} with varying metric proxy frequencies.
Each run will produce a .dat file containing the various runs. Then the plotting script will read these files to average and then plot the result.
It is important to note that the baseline is hardcoded in the python script from the wrapper overhead measurements. And thus if you rerun it you will
need to provide the average instrumented cost (first collumn in \texttt{lib.dat}) to match your machine. All the steps can be run by calling the \texttt{measure\_ovh.sh} script
which will take care of performing the various measurements before calling the plotting script to display the results, this scripts requires to have the metric proxy bindir in the PATH.

\subsection{Time needed for the workflow}

The process should not be too time consuming and results shall be obtained in minutes.

\subsection{The experiments and their results}

\paragraph*{\textbf{Figure 7:}} This figure presents the overhead of the proxy instrumentation in function of the measurement period. It shows that the longer the period the less perturbation is made to the run as the local proxy competes less with the running program. It shows that depending on the frequency, the overhead varies from 74 to 57 nanoseconds. Most of this overhead is linked to the lock protecting the counters. Artifacts associated with these results are provided in \texttt{proxy\_overhead/PAPER\_DATA}, you may enter this directory and regenerate the plot with \texttt{python3 ../plot\_ovh.py}.

%%%%%%%%%%%%%%%%%%%%%%%%%%%%%% Performance modeling
\section{Performance Modeling With Extra-P}
\label{sec:extra-p}
In this section, we closely examine how the profiles from the different application runs 
can be merged together to be used by Extra-P. This is necessary, as examining the scaling 
behavior requires more than just a single run. In the following, we first examine the workflow. 
Next, we look into how much time this workflow requires. Finally, we close up with a discussion 
on the importance of these experiments. 

\subsection{Workflow description}
After an application run finishes, the profiles are saved to a specified directory (usually in 
the home directory of the user). 
These runs are labeled by different entries including the command line arguments passed to SLURM. 
The first step required for generating performance models with Extra-P is to group the 
profiles into a single file. While Extra-P supports various file formats ( 
for example, cube files where each run can have a dedicated file); formats like JsonLines 
are ideal for our approach, as whenever a new profile is available, it can be appended to the same 
file. This can be done by the following call: 
\begin{lstlisting}[language=bash]
tau_profile_inspect -G id,.. -E out.jsonl -c 
\end{lstlisting}
For the command, "id" is replaced by the list of relevant profiles. Once this is done, "out.jsonl" 
is generated. 

The second step in this approach is to pass the file to Extra-P. 
Extra-P can be executed in a GUI or directly in a terminal. For the paper, 
we selected the first option. Next, the data can be loaded once the GUI is opened or 
directly using the command line interface. For the latter one (used in the paper), the call is:
\begin{lstlisting}[language=bash]
extrap-gui --scaling weak --json out.jsonl
\end{lstlisting}

This launches Extra-P, as shown in Figures~13-15 in the paper. After that, the relevant 
metric, along with the functions of interest, can be selected for examining their scaling. 


\subsection{Time needed for the workflow}
The described workflow in this section was done for Nek5000 and IMB-IO. For both cases, the 
time needed for grouping the profiles to a single JsonLines files and the model generation 
only consumed a few seconds. 

\subsection{The experiments and their results}
The results were covered in Section~6.4 of the submitted paper. We have shown three 
figures (Figure~13-15). In the different figures, we demonstrated new aspects related 
to the model generation with Extra-P.
\paragraph*{\textbf{Figure 13:}} This figure is concerned with the scaling behavior of IMB-IO. 
The exact configuration parameters were provided in the paper. The figure demonstrates the capability 
of our approach to feeding Extra-P with the required metrics to generate performance models. On the 
other hand, the generated performance models could be used for the optimization of resource management or 
in malleable decisions. The latter requires adaptations mentioned in the paper's last section.  

\paragraph*{\textbf{Figure 14:}}
In principle, this figure presents similar results as the previous one. The difference is that it is topped with 
two additional aspects: (1) While Figure~13 focused on metrics obtained by the metric proxy, this figure presents 
additional metrics obtained by strace ("pread64"), and (2) it illustrates a new metric in Extra-P, namely the number of bytes. 
Hence, we extended the capabilities of Extra-P to generate as well performance models in terms of bytes and not only time. 
Note that performance models in terms of function visits are also generated but not shown. 

\paragraph*{\textbf{Figure 15:}}
Unlike the previous two figures, this figure shows the performance model generated for Nek5000. 
Besides handling a different example which is in this case, a real application, this figure shows the 
scaling behavior of six selected strace functions regarding the number of ranks versus time. From the 
scaling behavior, one can depict that "ioctl" probably will perform worst compared to the other functions. 

\section{Conclusion}
In short, all the experiments are reproducible. As the collection of the metrics can be affected by the load 
and type of system used, the exact data might vary. If the experiments are repeated on the Turin cluster, 
we expect similar results as presented in the paper. However, to make the generation of the results, including 
the figures, perfectly reproducible, we provide the data we collected. 
This includes the JsonLines files to generate the performance models. Moreover, as we also provide the profiles for all experiments, 
the JsonLines files are as well reproducible. 
All profiles and files are publicly available under the following git repository: \url{https://github.com/besnardjb/always_on_monitoring_for_hpc}
\end{document}
